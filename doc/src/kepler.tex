There are some necessary corrections to do with the normal Hermite integrator scheme.  \begin{itemize} \item The BH needs to be fixed in the origin, this means that we need to
          exclude it from the normal algorithm loops.
          In this case we use a pre-processor directive called
          \texttt{INIT\_PARTICLE} which determinate if we include the particle $0$
          (The BH) or not.
    \item We need to define more pre-processor constants to handle the keplerian
          integration.
          \begin{itemize}
              \item The maximum iteration when solving the Kepler's equation
                    \texttt{KEPLER\_ITE}.
              \item The maximum error in the Eccentricity anomaly, in a elliptical
                    orbit. \texttt{DEL\_E}
              \item The maximum error in the Eccentricity anomaly, in a hyperbolic
                    orbit. \texttt{DEL\_E\_HYP}
              \item The maximum of steps when calculating the central time steps
                    distribution. \texttt{OSTEPS}
          \end{itemize}
    \item Before the Hermite prediction step, we need to perform the prediction
          of the active particles using the Keplerian approach explained below.
\end{itemize}

For the Keplerian prediction, we need to perform the following steps for every
active particle ($i$-particle) in a certain integration time ($t$),
to move it in a $dt$.
\setlength{\columnseprule}{0.4pt}
\begin{multicols}{2}
\begin{enumerate}
    \item Calculate relative position and velocity between the BH and
          an $i$-particle.
          \begin{align}
              \bs{r}_{k,i} &= \bs{r}_{i} - \bs{r}_{0} \\
              \bs{v}_{k,i} &= \bs{v}_{i} - \bs{v}_{0}
          \end{align}

    \item Calculate the angular momentum vector:
        \begin{align}
            \bs{j} &= \bs{r}_{k,i} \times \bs{v}_{k,i}
        \end{align}

    \item Calculate the Runge-Lenz-vector:
        \begin{align}
            \bs{e} &= \frac{\bs{v}_{k,i} \times \bs{j}}{G m} - \frac{\bs{r}}{|\bs{r}|}
        \end{align}

    \item Calculate the eccentricity
        \begin{align}
            e  &= |\bs{e}|
        \end{align}

    \item Calculate the eccentricity
        \begin{align}
            e  &= |\bs{e}|
        \end{align}

    \item Calculate the semi-major axis
        \begin{align}
            a &= \frac{\bs{j}^{2}}{G m |1 - e^{2}|}
        \end{align}

    \item Calculate the Frequency of the orbit
        \begin{align}
            \omega &= \sqrt{\frac{Gm}{a^{3}}}
        \end{align}

    \item Calculate the Semi-major vector
        \begin{align}
            \bs{a} &= \frac{a}{e}\bs{e}
        \end{align}

    \item Calculate the Semi-minor axis
        \begin{align}
            b &= a \sqrt{|1-e^{2}|}
        \end{align}

    \item Calculate the Semi-minor vector
        \begin{align}
            \bs{b} &= b \widehat{(\bs{j} \times \bs{e})}
        \end{align}
\end{enumerate}
\end{multicols}

Then, we continue depending of the eccentricity value.

\begin{itemize}
    % Elliptical orbit
    \item If $e < 1$,
        \begin{enumerate}
            \item Calculate the initial Eccentric Anomaly ($E_{0}$),
                \begin{align}
                    \cos{E_{0}} &= \frac{a-r}{e a} \\
                          E_{0} &= \arccos{\left(\frac{a-r}{e a}\right)}
                \end{align}

                If the value of $\frac{a-r}{e a}$ is $\geq 1$, we set it
                to $0$ and if it is $\leq -1$ we set it as $\pi$.
                In any other scenario, we keep the same value.

            \item If $|\bs{r}_{k,i} \cdot \bs{b}| < 0$,
                we set the $E_{0} = 2\pi - E_{0}$.

            \item Calculate the Mean Anomaly,
                \begin{align}
                    M &= (E_{0} - e \sin{E_{0}} ) + dt * \omega
                \end{align}
            \item Adjust M to be $< 2\pi$ if it is $\geq 2\pi$,
                \begin{align}
                    M &= M \mod 2\pi
                \end{align}

            \item Solving the Kepler equations to calculate the Eccentric Anomaly ($E$)
                using the Mean Anomaly ($M$) and Eccentricity ($e$). (See Appendix~\ref{ap:kepler-elliptical})
            \item We calculate the position and velocity constant,
                \begin{align}
                    r_{const} &= \cos{E} - e \\
                    v_{const} &= \frac{\omega}{1 - e \cos{E}}
                \end{align}

            \item If the eccentricity is greater than 0.99~\footnote{Based on
                  Löckmann and Baumgardt simulations criteria, this work better
                  with $0.99 < e < 1$ and $|E| < 1e-3$}.

            \begin{lstlisting}[language=C]
if (e > 0.99)
{
    e_tmp (E > 2.0 * PI - 1e-3) ? E - 2.0 * PI : E;

    if (e_tmp < 1e-3)
    {
        e_tmp *= e_tmp;
        ecc_const = (jx*jx + jy*jy + jz*jz)/(m0 * a * (1 + e));
        cos_const = -0.5 * e_tmp * (1 - e_tmp / 12.0 * (1 - e_tmp / 30.0));
        r_const = ecc_const + cos_const;
        v_const = w / (ecc_const - ecc * cos_const);
    }
}
            \end{lstlisting}

            \item We get the new position and velocity:
                \begin{align}
                    \bs{r}_{new} &= \bs{a}\cdot r_{const} + \bs{b} \cdot \sin(e) \\
                    \bs{v}_{new} &= (-\bs{a}\cdot \sin(e) + \bs{b}\cdot \cos(e)) \cdot v_{const}
                \end{align}
        \end{enumerate}


    % Hyperbolic/Parabolic orbit
    \item If $e \geq 1$,
        \begin{enumerate}
            \item Calculate the initial Eccentric Anomaly ($E_{0}$):
                \begin{align}
                    \cosh{E_{0}} &= \frac{a+r}{e a} \\
                          E_{0} &= \arccosh{\left(\frac{a-r}{e a}\right)}
                \end{align}

                If the value of $\frac{a+r}{e a}$ is $< 1$, we set it
                to $0$.
                (else if) In other case if $(\bs{r}\cdot\bs{v} < 0)$ we use $-\arccosh$ instead
                of $\arccosh$.
                (else) Finally, if the previous conditions are not valid, we use the previous
                value $\arccosh{\left(\frac{a-r}{e a}\right)}$.

            \item We calculate the Mean Anomaly ($M$).
                \begin{align}
                    M = e \sinh(E_{0}) - E_{0} + dt\omega
                \end{align}

            \item We Solve the keplerian equations using $e$, and $M$ to get the
                    new value of the Eccentricity Anomaly ($E$)~\ref{ap:kepler-hyperbollic}.

            \item Calculate the velocity constant:
                \begin{align}
                    v_{const} &= \frac{\omega}{e \cosh(E) - 1}
                \end{align}
%
            \item We get the new position and velocity:
                (Direction of $\bs{v}$ only)
                \begin{align}
                    \bs{r}_{new} &= \bs{a}\cdot (e - \cosh(E)) + \bs{b}\cdot \sinh(E) \\
                    \bs{v}_{new} &= (-\bs{a}\cdot \sinh(E) + \bs{b}\cdot \cosh(E)) \cdot v_{const}
                \end{align}
        \end{enumerate}
        \item The new predicted position and velocity of the particle will be,
            \begin{align}
                \bs{r}_{new} &+= \bs{r}_{BH}\\
                \bs{v}_{new} &+= \bs{v}_{BH}\\
            \end{align}
            Since in the first step we used the relative position within the BH.

\end{itemize}

After this procedure, we will have the new position and velocity
of a particle related to the BH interaction.
This steps happens only then the particle it is active ($i\in n_{act}$),
but \emph{What happen when the particle is not active? How do we include
the BH effect on his motion?}.

Since we use a split scheme, for non-active particles we use the normal
Hermite 4th predictor, but including the interaction with the BH in the previous
step in which the particle was active.

To calculate the direct interaction of the particle with the BH,
we obtain the acceleration and its derivatives with the following
formulas:

\begin{align}
    \bs{a}_{i,BH}       &= -G\frac{m_{0}}{r^{3}} \bs{r}\\
    \bs{a}_{i,BH}^{(1)} &= -G\frac{m_{0}}{r^{3}} \left(\bs{v} - 3 \frac{(\bs{v}\cdot\bs{r})\bs{r}}{r^{2}} \right) \\
    \bs{a}_{i,BH}^{(2)} &= -G\frac{m_{0}}{r^{3}} \left( \bs{a} - \frac{3}{r^{2}}
                                                    \left[ (\bs{v}\cdot\bs{r})
                                                        \left\{
                                                            2\bs{v} - 5 \frac{(\bs{v}\cdot\bs{r})\bs{r}}{r^{2}}
                                                        \right\}
                                                    \right]
                                                 \right) \\
    \bs{a}_{i,BH}^{(3)} &= -G\frac{m_{0}}{r^{3}} \left( \bs{a}^{(1)} - \frac{3}{r^{2}}
                                                    \left[
                                                        3 (\bs{v}\cdot\bs{r})\bs{a} + 3 (v^{2} + (\bs{r}\cdot \bs{a}))
                                                        \left\{ \bs{v} - 5 \frac{(\bs{v}\cdot\bs{r})\bs{r}}{r^{2}}       \right\} \nonumber \right.\right.\\
                                                        &\qquad {} \left.\left. +\left\{ 3 (\bs{v}\cdot\bs{a}) + (\bs{r}\cdot\bs{a}^{(1)}) \bs{r} \right\} +
                                                        \frac{(\bs{v}\cdot\bs{r})^{2}}{r^{2}}
                                                        \left\{-15 \bs{v} + 35\frac{(\bs{v}\cdot\bs{r}) \bs{r}}{r^{2}}  \right\}
                                                    \right]
                                                 \right)
\end{align}
